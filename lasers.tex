\subsection{Laser-based Range Sensors} \index{Laser-based Range Sensors}

There are three types of laser based range sensors in common use:
\begin{enumerate}
\item triangulation sensors,
\item phase modulation sensors and
\item time of flight sensors.
\end{enumerate}
These are discussed in more detail below.
There are also doppler and interference laser-based range
sensors, but these seem to not be in general use at the moment so we 
skip them here.
An excellent recent review of range sensors is by Blais  \cite{blais}.

The physical principles behind the three types of sensors discussed below do
not intrinsically depend on the use of a laser - for the most part any light 
source would work.
However, lasers are traditional because: 
1) they can easily generate bright beams with lightweight sources,
2) infrared beams can be used inobtrusively, 
3) they focus well to give narrow beams, 
4) single frequency sources allow easier rejection filtering of unwanted frequencies,
5) single frequency sources do not disperse from refraction as much as
   full spectrum sources,
6) semiconductor devices can more easily generate short pulses, etc.

One advantage of all three sensor types is that it is often
possible to acquire a
reflectance image registered with the range image.
By measuring the amount that the laser beam strength has reduced after
reflection from the target, one can estimate the reflectance of the
surface. This is only the reflectance at the single spectral frequency of the
laser, but, nonetheless, this gives useful information about the appearance
of the surface (as well as the shape as given by the range measurement).
(Three color laser systems \cite{Baribeau} similarly give registered RGB color 
images.)
Figure \ref{fig23.1} shows registered range and reflectance images from the same scene.

One disadvantage of all three sensor types is specular reflections.
The normal assumption is that the observed light is a diffuse reflection
from the surface.
If the observed surface is specular, such as polished metal or water, then
the source illumination may be reflected in unpredictable directions.
If any light eventually is detected by the receiver, then it is likely to
cause incorrect range measurements.
Specular reflections are also likely at the fold edges of surfaces.

A second problem is the laser `footprint'.
Because the laser beam has a finite width, when it strikes at the edge
of a surface, part of the beam may be actually lie on a more distant surface.
The consequences of this will depend on the actual sensor, but it commonly
results in range measurements that lie between the closer and more distant
surfaces.
These measurements thus suggest that there is surface where there really is 
empty space.
Some noise removal is possible in obvious empty space, but there may be
erroneous measurements that are so close to the true surface that they
are difficult to eliminate.

\subsection{Time of Flight Range Sensors}

\index{Time of flight range sensors} Time of flight range sensors are exactly that: they compute distance
by measuring the time that a pulse of light takes to travel from the 
source to the observed target and then to the detector (usually colocated
with the source).
In a sense, they are radar sensors that are based on light.
The travel time multiplied by the speed of light (in the given medium - space,
air or water and adjusted for the density and temperature of the medium) gives the
distance.
Laser-based time of flight range sensors are also caller LIDAR
(LIght Detection And Ranging) or LADAR (LAser raDAR) sensors.

Limitations on the accuracy of these sensors is based on the minimum 
observation time (and thus the minimum distance observable),
the temporal accuracy (or quantization) of the receiver and the temporal
width of the laser pulse.
%A minimum pulse width sensor \cite{} uses single photons and very sensitive
%detectors to get a range accuracy of xxx mm with an stand-off distance 
%of xxx cm. Multiple photon transmissions ({\it e.g.} 1000) are sampled to get 
%accurate measurements.

Many time of flight sensors used for local measurements have what is 
called an ``ambiguity interval'', for example 20 meters.
The sensor emits pulses of light periodically, and computes an average
target distance from the time of the returning pulses. 
To limit noise from reflections and simplify the detection electronics,
many sensors only accept signals that arrive within time $\Delta t$,
but this time window might also observe previous pulses reflected by 
more distant surfaces.
This means that a measurement $Z$ is ambiguous to the multiple of
$\frac{1}{2}c\Delta t$ because surfaces that are further away ({\it e.g.} $z$) than 
$\frac{1}{2}c\Delta t$ are recorded as $z {\rm mod} \frac{1}{2}c\Delta t$.
Thus distances on smooth surfaces can increase until they reach
$c\frac{\Delta t}{2}$ and then they become 0.
Typical values for $c\frac{\Delta t}{2}$ are 20-40m.
On smooth surfaces, such as a ground plane or road, an unwinding algorithm
can recover the true depth by assuming that the distances should be changing
smoothly.

Most time of flight sensors transmit only a single beam, thus range measurements
are only obtained from a single surface point.
Robotics applications usually need more information, so the range data 
is usually supplied as a vector of range to surfaces lying in a plane
(see Figure \ref{lineing}) or as an image (see Figure \ref{fig23.1}).
To obtain these denser representations, the laser beam is swept across the
scene.
Normally the beam is swept by a set of mirrors rather than moving 
the laser and detector themselves (mirrors are lighter and less prone 
to motion damage).
The most common technologies for this are using a stepper motor (for
program-based range sensing) or rotating or oscillating mirrors for
automatic scanning.
\begin{figure}[htb]
{\epsfxsize = 0.47\textwidth \epsfbox{BOOKFIGS/lineing.eps}}
\caption{Plot of ideal 1D range image of sample distance versus angle of measurement.
%\label{lineing}
}
\end{figure}

Typical ground-based time of flight sensors suitable for robotics
applications have a range of 10-100m, and an accuracy of 5-10mm.
The amount of the scene scanned will depend on the sweep rate of the 
mirrors and the pulse rate, but 1-25K points per second are typical.
Manufacturers of these sensors include Acuity, Sick, Mensi,
DeltaSphere and Cyrax.

Recently, a type of time-of-flight range sensor called the ``Flash LADAR''
has been developed.
The key innovation has been the inclusion of VLSI timing circuits at each
pixel of the sensor chip.
Thus, each pixel can measure the time at which a light pulse is observed
from the line of sight viewed by that pixel.
This allows simultaneous calculation of the range values at each pixel.
The light pulse now has to cover to whole portion of the scene that is
observed, so sensors typically use an array of infrared laser LEDs.
While spatial resolution is smaller than current cameras ({\it e.g.}
$64\times 64$, $160\times 124$, $128\times 128$), the data can be acquired
at video rates (30-50 fps), which provides considerable information 
usable for robot feedback.
Different sensor ranges have been reported, such as up to 5m \cite{anderson}
(with on the order of 5-50 cm standard deviation depending on
target distance and orientation)
or at 1.1 km \cite{stettner} (no standard deviation reported).

\subsection{Modulation Range Sensors}

\index{Modulation-based range sensors} Modulation-based range sensors are commonly of two types, where a
continuous laser signal is either amplitude or frequency modulated.
By observing the phase shift between the outgoing and return signals,
the signal transit time is estimated and from this the target distance.
As the signal phase repeats every $2 \pi$, these sensors also have
an ambiguity interval.


These sensors also produce a single beam that must be swept.
Scan ranges are typically 20-40m and accuracy of 5mm.
Figure \ref{fig23.1} was captured with a modulation sensor.

\subsection{Triangulation Range Sensors}


\index{Triangulation range sensors} Triangulation range sensors \cite{LeMoigne} are based on principles similar to the
stereo sensors discussed previously.
The key concept is illustrated in Figure \ref{lasertri}: a laser beam is projected
from one position onto the observed surface.
The light spot that this creates is observed by a sensor from a second
position.
Knowing the relative positions and orientations of the laser and sensor,
plus some simple trigonometry allows calculation of the 3D position
of the illuminated surface point.
The triangulation process is more accurate when the laser spot's
position is accurately estimated. About 0.1 pixel accuracy can be
normally achieved \cite{naidu}.
\begin{figure}[htb]
{\epsfxsize = 0.47\textwidth \epsfbox{BOOKFIGS/lastri.eps}}
\caption{Triangulation using a laser spot.
\label{lasertri}}
\end{figure}

Because the laser illumination can be controlled, this gives a number
of practical advantages:
\begin{enumerate}

\item A laser of known frequency ({\it e.g.} 733 nm) can be matched with
a very selective spectral filter of the same frequency ({\it e.g.} 5 
nm half width). This normally allows unique identification of the 
light spot as the filtering virtually eliminates all other bright
light, leaving the laser spot as the brightest spot in the image.

\item The laser spot can be reshaped with lenses or mirrors to create
multiple spots or stripes, thus allowing a measurement of multiple
3D points simultaneously. Stripes are commonly used because these can
be swept across the scene (see Fig \ref{swept} for an example) to observe a full
scene.
Other illumination patterns are also commonly used, such
as parallel lines, concentric circles, cross hairs and dot grids.
Commercial structured light pattern generators are available from
{\it e.g.} Lasiris or Edmunds Optics.

\begin{figure}[htb]
{\epsfxsize = 0.47\textwidth \epsfbox{BOOKFIGS/swept.eps}}
\caption{Swept laser plane covers a larger scene.
\label{swept}}
\end{figure}


\item The laser light ray can be positioned by mirrors under computer
control to selectively observe a given area, such as around a doorway
or potential obstacle, or an object that might be grasped.

\end{enumerate}



Disadvantages include:

\begin{enumerate}

\item Potential eye safety risks from the power of lasers, particularly
when invisible laser frequencies are used (commonly infrared).

\item Specular reflections from metallic and polished objects, which may cause
incorrect calculation of where the laser spot is illuminating, as in Figure 
\ref{specular} where the hypothesized surface lies behind the true surface.

\end{enumerate}
\begin{figure}
{\epsfxsize = 0.47\textwidth \epsfbox{BOOKFIGS/specular.eps}}
\caption{Incorrect estimated depths on specular surfaces
\label{specular}}
\end{figure}

\subsection{Example Sensors}

A typical phase modulation based range sensor is the Zohler and Frohlich 25200.
This is an expensive spherical scan sensor, observing a full 360 degrees
horizontally and 155 vertically. Each scan can acquire up to 20,000 3D points
horizontally and 8000 points vertically in about 100 seconds.
The accuracy of the scan depends on the distance to the target, but
4 mm is typical, with samples every 0.01 degrees (which samples every
1.7 mm at 10m distance).
The densely sensed data is typically used for 3D scene survey, modelling and
virtual reality reconstruction.

An example of a laser sensor that has been commonly used for robot navigation
is the SICK LMS 200.
This sensor sweeps a laser ray in a horizontal plane through an arc of 180
degrees, acquiring 720 range measurements in a plane in 0.05 seconds.
While a plane might seem limiting, the observed 3D data can be easily matched
to a stored (or previously learned) model of the environment at the
height of the scan.
This allows accurate estimation of the sensor's position in the scene
(and thus a mobile vehicle that carries the sensor).
Of course, this requires an environment without overhanging structures
that are not detectable at the scan height.
Another common configuration is to mount the sensor high on the vehicle 
with the scan plane tilted downwards. In this configuration the ground
in front of the vehicle is progressively scanned as the vehicle moves forward,
thus allowing detection of potential obstacles.

The Minolta 910 is a small triangulation scanner with accuracy of about 0.01 mm
in a range of up to 2m, for about $500^2$ points in 2.5 seconds.
It has commonly been used for small region scanning, such as for inspection
or part modelling, but can also be mounted in a fixed location to scan
a robot workcell. 
Observation of the workcell allows a variety of actions, such as
inspection of parts, location of dropped or bin delivered components,
or accurate part position determination.

More examples and details range sensors for robot navigation are presented
in Section \ref{ch31.3}.
